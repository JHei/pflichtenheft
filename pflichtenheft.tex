\documentclass[a4paper]{scrreprt}
 
\usepackage[german]{babel}
\usepackage[utf8]{inputenc}
\usepackage[T1]{fontenc}
\usepackage{ae}
\usepackage[bookmarks,bookmarksnumbered]{hyperref}
 
\begin{document}
 
\title{Pflichtenheft mit LaTeX}
\author{Karl Lorey}
\maketitle
 
\tableofcontents
\vspace{1cm}
Dies ist ein beispielhaftes Pflichtenheft in \LaTeX. Das Pflichtenheft
beschreibt in konkreter Form, wie der Auftragnehmer die Anforderungen des
Auftraggebers zu lösen gedenkt - das sogenannte wie und womit. Der Auftraggeber
beschreibt vorher im Lastenheft möglichst präzise die Gesamtheit der 
Forderungen - was er entwickelt oder produziert haben möchte. Erst wenn der 
Auftraggeber das Pflichtenheft akzeptiert, sollte die eigentliche Arbeit beim 
Auftragnehmer beginnen.
 
Quelle: \url{http://de.wikipedia.org/wiki/Pflichtenheft} und Lehrbuch der 
Objektmodellierung von Heide Balzert
 
Quellcode: \url{http://www.karllorey.de/}
 
\chapter{Zielbestimmung}
 
\section{Musskriterien}
Musskriterien: für das Produkt unabdingbare Leistungen, die in jedem Fall
erfüllt werden müssen
 
\section{Kannkriterien}
Kannkriterien: die Erfüllung ist nicht unbedingt notwendig, sollten nur
angestrebt werden, falls noch ausreichend Kapazitäten vorhanden sind.
 
\section{Abgrenzungskriterien}
Abgrenzungskriterien: diese Kriterien sollen bewusst nicht erreicht werden
 
\chapter{Einsatz}
 
\section{Anwendungsbereiche}
 
\section{Zielgruppen}
 
\section{Betriebsbedingungen}
Betriebsbedingungen: physikalische Umgebung des Systems, tägliche Betriebszeit,
ständige Beobachtung des Systems durch Bediener oder unbeaufsichtigter Betrieb
 
\chapter{Umgebung}
 
\section{Software}
Software: für Server und Client, falls vorhanden
 
\section{Hardware}
Hardware: für Server und Client getrennt
 
\section{Orgware}
Orgware: organisatorische Rahmenbedingungen
 
\chapter{Funktionalität}
Funktionalität: genaue und detaillierte Beschreibung der einzelnen
Produktfunktionen
 
\chapter{Daten}
Daten: langfristig zu speichernde Daten aus Benutzersicht
 
\chapter{Leistungen}
Leistungen: Anforderungen bezüglich Zeit und Genauigkeit
 
\chapter{Benutzungsoberfläche}
Benutzungsoberfläche: grundlegende Anforderungen, Zugriffsrechte
 
\chapter{Qualitätsziele}
 
\chapter{Anhang}
 
\end{document}